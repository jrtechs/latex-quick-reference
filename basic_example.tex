% this is a comment!!!
% these don't appear in your final rendered document


\documentclass{article} % Starts an article
\usepackage{amsmath} % Imports amsmath
\title{\LaTeX Introduction} % Title
% \LaTeX prints LaTeX all fancy like
\author{Jeffery Russell}

\begin{document} % Begins a document
  \maketitle % places document at start of file
  \LaTeX{} is a document preparation system for
  the \TeX{} typesetting program.

\section{Math}

	In \LaTeX, there are tons of options for writing math.
	Over time you will memorize a lot of them, however, TexMaker has a side panel with all the math symbols. Math can be done in-line with single dollar signs like this: $y = mx +b$. Isn't this cool! However, math can also be done in bigger blocks that you can then reference.

  % The following shows some math examples
  \begin{align}
  	\label{math1}
    E_0 &= mc^2 \\
    E &= \frac{mc^2}{\sqrt{1-\frac{v^2}{c^2}}}
  \end{align} 
  
 Look at equation \ref{math1}. Isn't that a cool function! Math blocks can also be constructed using the double dollar sign. But, these don't get a reference number associated with them.
 
 
$$
\begin{bmatrix}
1 & 3\\
7 & 5
\end{bmatrix} \odot
\begin{bmatrix}
6 & 8\\
4 & 2
\end{bmatrix}\\*
$$



Cool right?



\end{document}